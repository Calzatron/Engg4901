\documentclass[12pt,openany,a4paper]{book}
\usepackage{graphics}	% if you want encapsulated PS figures.
\usepackage[utf8]{inputenc}
\usepackage{hyperref}
\usepackage{textcomp}
\usepackage{listings}
\usepackage{xcolor}
\usepackage{blindtext}
\usepackage{enumitem}
\usepackage{bm}
\usepackage{courier}
\usepackage{amssymb}
\usepackage{mathtools}
\usepackage{mathrsfs,amsmath}   %The amsmath package
\definecolor{mygreen}{rgb}{0,0.6,0}
\definecolor{mygray}{rgb}{0.5,0.5,0.5}
\definecolor{mymauve}{rgb}{0.58,0,0.82}

\DeclareMathSizes{10}{10}{10}{10}

\lstset{ %
  backgroundcolor=\color{white},   % choose the background color; you must add \usepackage{color} or \usepackage{xcolor}
  basicstyle=\footnotesize,        % the size of the fonts that are used for the code
  breakatwhitespace=true,         % sets if automatic breaks should only happen at whitespace
  breaklines=true,                 % sets automatic line breaking
  captionpos=b,                    % sets the caption-position to bottom
  commentstyle=\color{mygreen},    % comment style
  deletekeywords={...},            % if you want to delete keywords from the given language
  escapeinside={\%*}{*)},          % if you want to add LaTeX within your code
  extendedchars=true,              % lets you use non-ASCII characters; for 8-bits encodings only, does not work with UTF-8
  frame=single,	                   % adds a frame around the code
  keepspaces=true,                 % keeps spaces in text, useful for keeping indentation of code (possibly needs columns=flexible)
  keywordstyle=\color{blue},       % keyword style
  language=Matlab,                 % the language of the code
  otherkeywords={*,...},           % if you want to add more keywords to the set
  numbers=none,                    % where to put the line-numbers; possible values are (none, left, right)
  numbersep=5pt,                   % how far the line-numbers are from the code
  numberstyle=\tiny\color{mygray}, % the style that is used for the line-numbers
  rulecolor=\color{black},         % if not set, the frame-color may be changed on line-breaks within not-black text (e.g. comments (green here))
  showspaces=false,                % show spaces everywhere adding particular underscores; it overrides 'showstringspaces'
  showstringspaces=false,          % underline spaces within strings only
  showtabs=false,                  % show tabs within strings adding particular underscores
  stepnumber=2,                    % the step between two line-numbers. If it's 1, each line will be numbered
  stringstyle=\color{mymauve},     % string literal style
  tabsize=4,	                   % sets default tabsize to 2 spaces
  title=\lstname,                   % show the filename of files included with \lstinputlisting; also try caption instead of title
  upquote=true,
}
\renewcommand{\lstlistingname}{Script}



\newcommand\numberthis{\addtocounter{equation}{1}\tag{\theequation}}


% If you use a macro file called macros.tex :
% \input{macros}
% Note: The present document has its macros built in.

% Number subsections but not subsubsections:
\setcounter{secnumdepth}{2}
% Show subsections but not subsubsections in table of contents:
\setcounter{tocdepth}{2}

\pagestyle{headings}		% Chapter on left page, Section on right.
\raggedbottom

\setlength{\topmargin}		{-5mm}  %  25-5 = 20mm
\setlength{\oddsidemargin}	{10mm}  % rhs page inner margin = 25+10mm
\setlength{\evensidemargin}	{0mm}   % lhs page outer margin = 25mm
\setlength{\textwidth}		{150mm} % 35 + 150 + 25 = 210mm
\setlength{\textheight}		{240mm} % 

\renewcommand{\baselinestretch}{1.2}	% Looks like 1.5 spacing.

% Stop figure/tables smaller than 3/4 page from appearing alone on a page:
\renewcommand{\textfraction}{0.25}
\renewcommand{\topfraction}{0.75}
\renewcommand{\bottomfraction}{0.75}
\renewcommand{\floatpagefraction}{0.75}

% THEOREM-LIKE ENVIRONMENTS:
\newtheorem{defn}	{Definition}	% cf. \dfn for cross-referencing
\newtheorem{theorem}	{Theorem}	% cf. \thrm for cross-referencing
\newtheorem{lemma}	{Lemma}		% cf. \lem for cross-referencing

% AIDS TO CROSS-REFERENCING (All take a label as argument):
\newcommand{\eref}[1] {(\ref{#1})}		% (...)
\newcommand{\eq}[1]   {Eq.\,(\ref{#1})}		% Eq.~(...)
\newcommand{\eqs}[2]  {Eqs.~(\ref{#1}) and~(\ref{#2})}
\newcommand{\dfn}[1]  {Definition~\ref{#1}}	% Definition~...
\newcommand{\thrm}[1] {Theorem~\ref{#1}}	% Theorem~...
\newcommand{\lem}[1]  {Lemma~\ref{#1}}		% Lemma~...
\newcommand{\fig}[1]  {Fig.\,\ref{#1}}		% Fig.~...
\newcommand{\tab}[1]  {Table~\ref{#1}}		% Table~...
\newcommand{\chap}[1] {Chapter~\ref{#1}}	% Chapter~...
\newcommand{\secn}[1] {Section~\ref{#1}}	% Section~...
\newcommand{\ssec}[1] {Subsection~\ref{#1}}	% Subsection~...

% AIDS TO FORMATTING:
\newcommand{\teq}[1]	{\mbox{$#1$}}	% in-Text EQuation (unbreakable)
\newcommand{\qed}	{\hspace*{\fill}$\bullet$}	% end of proof

% MATHEMATICAL TEMPLATES:
% Text or math mode:
\newcommand{\half}	{\ensuremath{\frac{1}{2}}}	% one-half
\newcommand{\halftxt}	{\mbox{$\frac{1}{2}$}}	  	% one-half, small
% Math mode only:
% N.B. Parentheses are ROUND; brackets are SQUARE!
\newcommand{\oneon}[1]	{\frac{1}{#1}}		  % reciprocal
\newcommand{\pow}[2]	{\left({#1}\right)^{#2}}  % Parenthesized pOWer
\newcommand{\bow}[2]	{\left[{#1}\right]^{#2}}  % Bracketed pOWer
\newcommand{\evalat}[2]	{\left.{#1}\right|_{#2}}  % EVALuated AT with bar
\newcommand{\bevalat}[2]{\left[{#1}\right]_{#2}}  % Bracketed EVALuated AT
% Total derivatives:
\newcommand{\sdd}[2]	{\frac{d{#1}}{d{#2}}}		    % Short
\newcommand{\sqdd}[2]	{\frac{d^2{#1}}{d{#2}^2}}	    % 2nd ("SQuared")
\newcommand{\ldd}[2]	{\frac{d}{d{#1}}\left({#2}\right)}  % Long paren'ed
\newcommand{\bdd}[2]	{\frac{d}{d{#2}}\left[{#2}\right]}  % long Bracketed
% Partial derivatives (same sequence as for total derivatives):
\newcommand{\sdada}[2]	{\frac{\partial {#1}}{\partial {#2}}}
\newcommand{\sqdada}[2]	{\frac{\partial ^{2}{#1}}{\partial {#2}^{2}}}
\newcommand{\ldada}[2]	{\frac{\partial}{\partial {#1}}\left({#2}\right)}
\newcommand{\bdada}[2]	{\frac{\partial}{\partial {#1}}\left[{#2}\right]}
\newcommand{\da}	{\partial}

% ORDINAL NUMBERS:
\newcommand{\ith}	{\ensuremath{i^{\rm th}}}
\newcommand{\jth}	{\ensuremath{j^{\rm th}}}
\newcommand{\kth}	{\ensuremath{k^{\rm th}}}
\newcommand{\lth}	{\ensuremath{l^{\rm th}}}
\newcommand{\mth}	{\ensuremath{m^{\rm th}}}
\newcommand{\nth}	{\ensuremath{n^{\rm th}}}

% SINUSOIDAL TIME AND SPACE-DEPENDENCY FACTORS:
\newcommand{\ejot}	{\ensuremath{e^{j\omega t}}}
\newcommand{\emjot}	{\ensuremath{e^{-j\omega t}}}

% UNITS (TEXT OR MATH MODE, WITH LEADING PADDING SPACE IF APPLICABLE):
% NB: These have not been tested since being modified for LaTeX2e.
\newcommand{\pack}	{\hspace{-0.08em}}
\newcommand{\Pack}	{\hspace{-0.12em}}
\newcommand{\mA}	{\ensuremath{\rm\,m\pack A}}
\newcommand{\dB}	{\ensuremath{\rm\,d\pack B}}
\newcommand{\dBm}	{\ensuremath{\rm\,d\pack B\pack m}}
\newcommand{\dBW}	{\ensuremath{\rm\,d\pack B\Pack W}}
\newcommand{\uF}	{\ensuremath{\rm\,\mu\pack F}}
\newcommand{\pF}	{\ensuremath{\rm\,p\pack F}}
\newcommand{\nF}	{\ensuremath{\rm\,n\pack F}}
\newcommand{\uH}	{\ensuremath{\rm\,\mu\pack H}}
\newcommand{\mH}	{\ensuremath{\rm\,m\pack H}}
\newcommand{\Hz}	{\ensuremath{\rm\,H\pack z}}
\newcommand{\kHz}	{\ensuremath{\rm\,k\pack H\pack z}}
\newcommand{\MHz}	{\ensuremath{\rm\,M\pack H\pack z}}
\newcommand{\GHz}	{\ensuremath{\rm\,G\pack H\pack z}}
\newcommand{\J}		{\ensuremath{\rm\,J}}
\newcommand{\kg}	{\ensuremath{\rm\,k\pack g}}
\newcommand{\K}		{\ensuremath{\rm\,K}}
\newcommand{\m}		{\ensuremath{\rm\,m}}
\newcommand{\cm}	{\ensuremath{\rm\,cm}}
\newcommand{\km}	{\ensuremath{\rm\,k\pack m}}
\newcommand{\mm}	{\ensuremath{\rm\,m\pack m}}
\newcommand{\nm}	{\ensuremath{\rm\,n\pack m}}
\newcommand{\um}	{\ensuremath{\rm\,\mu m}}
\newcommand{\Np}	{\ensuremath{\rm\,N\pack p}}
\newcommand{\s}		{\ensuremath{\rm\,s}}
\newcommand{\ms}	{\ensuremath{\rm\,m\pack s}}
\newcommand{\us}	{\ensuremath{\rm\,\mu s}}
\newcommand{\V}		{\ensuremath{\rm\,V}}
\newcommand{\mV}	{\ensuremath{\rm\,m\Pack V}}
\newcommand{\W}		{\ensuremath{\rm\,W}}
\newcommand{\mW}	{\ensuremath{\rm\,m\Pack W}}
\newcommand{\ohm}	{\ensuremath{\rm\,\Omega}}
\newcommand{\kohm}	{\ensuremath{\rm\,k\Omega}}
\newcommand{\Mohm}	{\ensuremath{\rm\,M\Omega}}
\newcommand{\degs}	{\ensuremath{\rm^{\circ}}}

% LaTeX run-time type-in command:
%
% \typein{Enter \protect\includeonly{...} command (or just type RETURN):}
%
% Uncommenting this command makes LaTeX prompt you for the \includeonly
% list.  At the prompt
%
%	\@typein=
%
% you type
%
%	\includeonly{chap1,chap2}
%
% to include the files chap1.tex and chap2.tex and omit any others.
% To include every \include file, just hit RETURN.
% If you are running LaTeX from xtexsh, you may need to click the mouse
% in the LaTeX window to position the cursor at the \@typein prompt.

\begin{document}

\frontmatter
% By default, frontmatter has Roman page-numbering (i,ii,...).

\begin{titlepage}
\renewcommand{\baselinestretch}{1.0}
\begin{center}
\vspace*{35mm}
\Huge\bf
		Virtual Robotics\\
		and Experimental\\
		Platform\\
\vspace{20mm}
\large\sl
		by\\
		Callum Rohweder
		\medskip\\
\rm
		School of Information Technology and Electrical Engineering,\\
		The University of Queensland.\\
\vspace{30mm}
		Submitted for the degree of\\
		Bachelor of Engineering
		\smallskip\\
\normalsize
		in the field of Mechatronics.
		\medskip\\
\large
		June \& 2018.		
\end{center}
\end{titlepage}

\cleardoublepage

\begin{flushright}
	ADDRESS LINE 1\\
	ADDRESS LINE 2\\
	Tel.\ (07) nnnn nnnn\\
	\medskip
	\today
\end{flushright}
\begin{flushleft}
  Prof Paul Strooper\\
  Head of School\\
  School of Information Technology and Electrical Engineering\\
  The University of Queensland\\
  St Lucia, Q 4072\\
  \bigskip\bigskip
  Dear Professor Strooper,
\end{flushleft}

In accordance with the requirements of the degree of Bachelor of
Engineering in the division of 
Electrical Engineering,
Electrical and Biomedical Engineering,
Electrical and Computer Engineering,
Software Engineering,
Mechatronic Engineering,
I present the
following thesis entitled ``\ldots''.  This work was performed [in
partnership with Mr/Ms \ldots\ and] under the supervision of
Mr/Ms/Dr/A/Prof./Prof.~\ldots.

I declare that the work submitted in this thesis is my own, except as
acknowledged in the text and footnotes, and has not been previously
submitted for a degree at The University of Queensland or any other
institution.

\begin{flushright}
	Yours sincerely,\\
	\medskip
	\emph{Author's Signature}\\
	\medskip
	AUTHOR'S NAME.
\end{flushright}

\cleardoublepage

% Dedication (if you want it):
\vspace*{70mm}
\begin{center}
\renewcommand{\baselinestretch}{1.0}
\sl
	To \ldots
\end{center}

\chapter{Acknowledgments}

Acknowledge your supervisor, preferably with a few short and specific
statements about his/her contribution to the content and direction of
the project.  If you collaborated with another student, acknowledge
your partner's contribution, including any parts of the thesis of
which s/he was the principal author or co-author; this information can
be duplicated in footnotes to the chapters or sections to which your
partner has contributed.  Briefly describe any assistance that you
received from technical or administrative staff.  Support of family
and friends may also be acknowledged, but avoid sentimentality---or
hide it in the dedication.

\cleardoublepage

\chapter{Abstract}

% Notice that all \include files are chapters -- a logical division.
% But not all chapters are \include files; some chapters are short
% enough to be in-lined in the main file.

This document is a skeleton thesis for 4th-year students.  The
printable versions (\texttt{skel.dvi, skel.ps, skel.pdf})
show the structure of a typical thesis with some notes on the content
and purpose of each part.  The notes are meant to be informative but
not necessarily illustrative; for example, this paragraph is not
really an abstract, because it contains information not found
elsewhere in the document.  The \LaTeXe\ source file
(\texttt{skel.tex}) contains some non-printing comments giving
additional information for students who wish to typeset their theses
in \LaTeX.  You can download the source, edit out the unwanted
material, insert your own frontmatter and bibliographic entries, and
in-line or \verb+\include{}+ your own chapter files.  Of course the
content of a particular thesis will influence the form to a large
extent.  Hence this document should not be seen as an attempt to force
every thesis into the same mold.  If in doubt about the structure of
your thesis, seek advice from your supervisor.

\tableofcontents

\listoffigures
\addcontentsline{toc}{chapter}{List of Figures}

\listoftables
\addcontentsline{toc}{chapter}{List of Tables}

% If file los.tex begins with ``\chapter{List of Symbols}'':
% \include{los}

\cleardoublepage

\mainmatter
% By default, mainmatter has Arabic page-numbering (1,2,...).


% Chapters may be \include files, each beginning with a line like
%
%	\chapter{Title of chapter}
%
% e.g. if two chapter files were called intro.tex and theory.tex,
% we would say
%
%	\include{intro}
%	\include{theory}

\chapter{Introduction}

The introductory chapter describes the importance of the field and the
scope and significance of your project.  It usually ends with an
overview of the remainder of the thesis.

Notice that Arabic page numbering begins with Chapter 1.  Preceding
pages (known as ``frontmatter'') have Roman numbering.  The
\texttt{book} document class in \LaTeX\ follows this numbering
convention by default (see Lamport~\cite{lamport}, p.\,80).

\chapter{Literature review / prior art}

You will need to review previous work in the field, which may include
books and papers (``literature''), patents and commercial products
(``prior art''), and earlier work in your Department.  This
information is usually (but not always) collected in a single chapter,
whose title should preferably be more specific and interesting than
the one above.

\chapter{Theory}

A scientific paper is likely to be read by people who are not
specialists in the same field as the author(s), but who nevertheless
may need to use the results of the paper in their own fields.
Similarly, the examiners of your thesis will probably include at least
one academic who does not teach or conduct research in the subject
area of your thesis.  In an early chapter of your thesis, therefore,
you should quote any theoretical results which are necessary for the
understanding of later chapters.  Examiners who are not specialists in
your area will know whether you have given sufficient theoretical
information.  They will also know whether you have insulted their
status by presenting material which is familiar to every
half-competent graduate in every field of ECE.

\chapter{Methodology, procedure, design, etc.}

This may be one chapter or several.  Again, titles should be more
informative than the above.

You will almost certainly need diagrams to clarify your meaning.  The
\LaTeXe\ \texttt{graphics} package allows the inclusion of PostScript
graphics, as in \fig{flr1}.  The inclusion of \LaTeX\ \texttt{picture}
graphics, as in \fig{fzsys}, requires no auxiliary packages and allows
the mathematical formatting features of \LaTeX\ to be used in
diagrams; but the \texttt{picture} files, unlike PostScript files,
usually require manual editing.





\chapter{Results and discussion \ldots}

\ldots\ or perhaps the discussion should be a separate chapter.

In any case, you will probably need to include tabulated results.
\tab{tf2} illustrates the use of various \LaTeX\ environments to
include a computer printout (plain text file) in a document.  The
\texttt{verbatim} environment, which encloses the formatted text, is
also useful for program listings.

\begin{table}\renewcommand{\baselinestretch}{1.0}
\caption{\sl Fraction of air volume involved in heat exchange for
second mode (right column) vs.\ filling factor (left column).  The
plain-text headings represent $f$, $m$, $\mu_2$ and $f_2$.}
\label{tf2}

\begin{center}
\begin{minipage}[c]{2.85in}\small\normalsize
\begin{verbatim}

 f(%)     m         mu2     f2(%)

 0.016   80.00    0.05400   4.874
 0.031   56.57    0.07732   5.438
 0.062   40.00    0.11103   6.125
 0.125   28.28    0.16001   6.970
 0.250   20.00    0.23175   8.020
 0.500   14.14    0.33799   9.329
 1.000   10.00    0.49789  10.967
 2.000    7.07    0.74444  13.008
 4.000    5.00    1.13919  15.525
 8.000    3.54    1.81095  18.568

19.237    2.28    3.61958  23.174
37.180    1.64    7.28635  27.094
57.392    1.32   14.63631  29.813
74.316    1.16   29.35160  31.453
85.734    1.08   58.79364  32.360
\end{verbatim}
\end{minipage}
\end{center}
\end{table}

\chapter{Conclusions}

\section{Summary and conclusions}

\section{Possible future work}

\appendix

% Chapters after the \appendix command are lettered, not numbered.
% Setting apart the appendices in the table of contents is awkward:

\newpage
\addcontentsline{toc}{part}{Appendices}
\mbox{}
\newpage

% The \mbox{} command between two \newpage commands gives a blank page.
% In the contents, the ``Appendices'' heading is shown as being on this
% blank page, which is the page before the first appendix.  This stops the
% first appendix from be listed ABOVE the word ``Appendices'' in the
% table of contents.

% \include appendix chapters here.

\chapter{Dummy appendix}

Appendices are useful for supplying necessary details or explanations
which do not seem to fit into the main text, perhaps because they are
too long and would distract the reader from the central argument.
Appendices are also used for program listings.

Notice that appendices are ``numbered'' with capital letters, not
numerals.  When the \verb+\appendix+ command in
\LaTeX~\cite[p.\,175]{lamport} is used with the \texttt{book} document
class, it causes subsequent chapters to be treated as appendices.

\chapter{Program listings}

\section{First program}

Some initial explanatory notes may precede the listing.

\section{Second program}

\section{Etc.}

\chapter{Companion disk}

If you wish to make some computer files available to your examiners,
you can list and describe the files here.  The files can be supplied
on a disk and inserted in a pocket fixed to the inside back cover.

The disk will not be needed if you can specify a URL from which the
files can be downloaded.

\cleardoublepage

\begin{thebibliography}{99}
\addcontentsline{toc}{chapter}{Bibliography}
\bibitem{lamport} L.~Lamport, \emph{\LaTeX: A Document Preparation
System}, 2nd ed. (Addison-Wesley, 1994).
\bibitem{LABEL2} REFERENCE 2
\bibitem{ETC.} Etc.
\end{thebibliography}

\end{document}